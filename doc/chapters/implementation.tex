Projekt je implementiran u programskom jeziku C++, programski kod dostupan je u datoteci \emph{src}. Kod koji implementira RRR strukturu nalazi se u datoteci \emph{src/rrr}, kod za binarno stablo valića u datoteci \emph{src/wavelet}, dok je kod koji sadrži pomoćne funkcije unutar datoteke \emph{src/utilty} te kod koji sadrži dijeljene definicije unutar \emph{src/shared}. Projekt je definiran \emph{Cmake} alatom koji služi za generiranje projekata za razna interaktivna razvijateljska sučelja za C++.

\subsubsection{RRRSequence}
Ovaj razred predstavlja RRR strukturu, koja se koristi za spremanje čvorova stabla valića u kodiranom ili ne kodiranom obliku. Razred sadrži metode za izvođenje upita rank0, rank1, select0, select1, access. Postoje dvije implementacije ovog razreda, jedna je pomoću kodiranja i druga bez kodiranja opisanog u prethodnom poglavlju. Implementacija bez kodiranja se nalazi na glavnoj grani git repozitroija, dok je implementacija s kodiranjem na grani feature/packing-unpacking-sequence istog repozitorija.

\subsubsection{RRRTable}
Ovaj razred je pomoćna struktura koja se koristi kod izgradnje i izvršavanja upita nad RRR strukturama. Sadrži metode za dohvat ranka0 i ranka1 na $i$-tom indeksu unutar nekog bloka, odmaka za dani rank i blok te metodu za dohvat broja bitova odmaka za dani rank.

\subsubsection{WaveletTree}
Ovaj razred predstavlja binarno stablo valića te sadrži metode za vršenje rank, select i access upita.

\subsubsection{WaveletNode}
Ovaj razred predstavlja čvor binarnog stabla valića  sadržava referencu na RRR strukturu stvorenu prilikom izgradnje stabla. Također sadrži metode za vršenje rank, select i access upita nad.

\subsubsection{bioinf\_utility}
Kod koji se koristi kod izvođenja glavnog programa za čitanje datoteka i dohvat memorije za pojedine operacijske sustave.

\subsubsection{common}
Header koji sadrži sve podatkovne tipove definirane za korištenje unutar implementacije strukture.

\subsubsection{main}
Glavna ulazna točka programa, sadrži metode za mjerenje memorije i vremena.