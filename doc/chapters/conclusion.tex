Pohrana i obrada ulaznih sekvenci velike duljine predstavlja pravi izazov jer u isto vrijeme želimo smanjiti potrošnju memorije i smanjiti utjecaj na performanse obrade istih. Taj problem je posebno izražen prilikom analize genetskih sekvenci, stoga je od iznimne važnosti u području bioinformatike. U ovom radu, kao struktura koja bi riješila gore navedeni problem, prikazana je implementacija statičke strukture binarnog stabla valića kao RRR strukture.

U radu je opisana teorijska podloga RRR strukture te binarnog stabla valića. Također opisana su i dva moguća pristupa prilikom izgradnje RRR strukture: kodirani i nekodirani. Kodirani pristup smanjuje memorijsko zauzeće uz mali utjecaj na performanse prilikom obavljanja upita.

Na kraju rada dani su i rezultati mjerenja i analize strukture na stvarnim i umjetnim podacima. Prava moć, to jest skalabilnost, RRR strukture vidljiva je prilikom mjerenja prosječnog vremena izvršavanja upita, gdje su se za jako velik porast duljine ulazne sekvence, i preko nekoliko milijuna znakova, vremena upita tek nešto blago linearno povećala. Memorijski rast je ipak bio nešto veći, no to je zbog same strukture stabla valića. Kao što je već spomenuto, stablo valića je statička struktura te kao takva nema mogućnost dinamičke izmjene podataka, već bi se svaki put trebalo izgraditi novo stablo, no u području bioinformatike to nam i nije neki problem jer u većini slučajeva želimo samo vršiti upite nad trenutnim podacima.