U području bioinformatike mogu se susresti sekvence znakova velikih duljina koje je potrebno obraditi ili izvršavati određene upite nad istima, poput pronalaska broja pojavljivanja određenog znaka do danog indeksa i sličnih. Slijedno obrađivanje navedenih sekvenci bi bilo vrlo sporo i neučinkovito te se poseže za nešto bržim, no i kompliciranijim rješenjima. Jedan od predstavnika je i binarno stablo valića kao RRR struktura koja je obrađena u ovom projektu.

Zadatak ovog projekta je bila izgradnja gore navedene strukture, provedba testiranja nad sintetičkim i stvarnim podacima u vidu brzine izgradnje stabla, potrošnje memorije te prosječnog vremena izvršavanja upita \emph{rank}, \emph{select} i \emph{access}. U nastavku dokumenta je prikazan opis RRR strukture, stabla valića te implementacije istih. Na kraju su dani rezultati i usporedba s onima od prošlogodišnjeg projekta \cite{breberic}.
